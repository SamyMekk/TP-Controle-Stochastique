% Options for packages loaded elsewhere
\PassOptionsToPackage{unicode}{hyperref}
\PassOptionsToPackage{hyphens}{url}
\PassOptionsToPackage{dvipsnames,svgnames,x11names}{xcolor}
%
\documentclass[
  10,
  letterpaper,
  DIV=11,
  numbers=noendperiod]{scrreport}

\usepackage{amsmath,amssymb}
\usepackage{iftex}
\ifPDFTeX
  \usepackage[T1]{fontenc}
  \usepackage[utf8]{inputenc}
  \usepackage{textcomp} % provide euro and other symbols
\else % if luatex or xetex
  \usepackage{unicode-math}
  \defaultfontfeatures{Scale=MatchLowercase}
  \defaultfontfeatures[\rmfamily]{Ligatures=TeX,Scale=1}
\fi
\usepackage{lmodern}
\ifPDFTeX\else  
    % xetex/luatex font selection
\fi
% Use upquote if available, for straight quotes in verbatim environments
\IfFileExists{upquote.sty}{\usepackage{upquote}}{}
\IfFileExists{microtype.sty}{% use microtype if available
  \usepackage[]{microtype}
  \UseMicrotypeSet[protrusion]{basicmath} % disable protrusion for tt fonts
}{}
\makeatletter
\@ifundefined{KOMAClassName}{% if non-KOMA class
  \IfFileExists{parskip.sty}{%
    \usepackage{parskip}
  }{% else
    \setlength{\parindent}{0pt}
    \setlength{\parskip}{6pt plus 2pt minus 1pt}}
}{% if KOMA class
  \KOMAoptions{parskip=half}}
\makeatother
\usepackage{xcolor}
\usepackage[lmargin=2cm,rmargin=2cm,tmargin=2cm,bmargin=3cm]{geometry}
\setlength{\emergencystretch}{3em} % prevent overfull lines
\setcounter{secnumdepth}{5}
% Make \paragraph and \subparagraph free-standing
\makeatletter
\ifx\paragraph\undefined\else
  \let\oldparagraph\paragraph
  \renewcommand{\paragraph}{
    \@ifstar
      \xxxParagraphStar
      \xxxParagraphNoStar
  }
  \newcommand{\xxxParagraphStar}[1]{\oldparagraph*{#1}\mbox{}}
  \newcommand{\xxxParagraphNoStar}[1]{\oldparagraph{#1}\mbox{}}
\fi
\ifx\subparagraph\undefined\else
  \let\oldsubparagraph\subparagraph
  \renewcommand{\subparagraph}{
    \@ifstar
      \xxxSubParagraphStar
      \xxxSubParagraphNoStar
  }
  \newcommand{\xxxSubParagraphStar}[1]{\oldsubparagraph*{#1}\mbox{}}
  \newcommand{\xxxSubParagraphNoStar}[1]{\oldsubparagraph{#1}\mbox{}}
\fi
\makeatother


\providecommand{\tightlist}{%
  \setlength{\itemsep}{0pt}\setlength{\parskip}{0pt}}\usepackage{longtable,booktabs,array}
\usepackage{calc} % for calculating minipage widths
% Correct order of tables after \paragraph or \subparagraph
\usepackage{etoolbox}
\makeatletter
\patchcmd\longtable{\par}{\if@noskipsec\mbox{}\fi\par}{}{}
\makeatother
% Allow footnotes in longtable head/foot
\IfFileExists{footnotehyper.sty}{\usepackage{footnotehyper}}{\usepackage{footnote}}
\makesavenoteenv{longtable}
\usepackage{graphicx}
\makeatletter
\newsavebox\pandoc@box
\newcommand*\pandocbounded[1]{% scales image to fit in text height/width
  \sbox\pandoc@box{#1}%
  \Gscale@div\@tempa{\textheight}{\dimexpr\ht\pandoc@box+\dp\pandoc@box\relax}%
  \Gscale@div\@tempb{\linewidth}{\wd\pandoc@box}%
  \ifdim\@tempb\p@<\@tempa\p@\let\@tempa\@tempb\fi% select the smaller of both
  \ifdim\@tempa\p@<\p@\scalebox{\@tempa}{\usebox\pandoc@box}%
  \else\usebox{\pandoc@box}%
  \fi%
}
% Set default figure placement to htbp
\def\fps@figure{htbp}
\makeatother
% definitions for citeproc citations
\NewDocumentCommand\citeproctext{}{}
\NewDocumentCommand\citeproc{mm}{%
  \begingroup\def\citeproctext{#2}\cite{#1}\endgroup}
\makeatletter
 % allow citations to break across lines
 \let\@cite@ofmt\@firstofone
 % avoid brackets around text for \cite:
 \def\@biblabel#1{}
 \def\@cite#1#2{{#1\if@tempswa , #2\fi}}
\makeatother
\newlength{\cslhangindent}
\setlength{\cslhangindent}{1.5em}
\newlength{\csllabelwidth}
\setlength{\csllabelwidth}{3em}
\newenvironment{CSLReferences}[2] % #1 hanging-indent, #2 entry-spacing
 {\begin{list}{}{%
  \setlength{\itemindent}{0pt}
  \setlength{\leftmargin}{0pt}
  \setlength{\parsep}{0pt}
  % turn on hanging indent if param 1 is 1
  \ifodd #1
   \setlength{\leftmargin}{\cslhangindent}
   \setlength{\itemindent}{-1\cslhangindent}
  \fi
  % set entry spacing
  \setlength{\itemsep}{#2\baselineskip}}}
 {\end{list}}
\usepackage{calc}
\newcommand{\CSLBlock}[1]{\hfill\break\parbox[t]{\linewidth}{\strut\ignorespaces#1\strut}}
\newcommand{\CSLLeftMargin}[1]{\parbox[t]{\csllabelwidth}{\strut#1\strut}}
\newcommand{\CSLRightInline}[1]{\parbox[t]{\linewidth - \csllabelwidth}{\strut#1\strut}}
\newcommand{\CSLIndent}[1]{\hspace{\cslhangindent}#1}

\newcommand{\bx}{\mathbf{x}}
\newcommand{\bt}{\boldsymbol{\theta}}
\newcommand{\dkl}{\mathrm{d}_{\mathrm{KL}}}
\newcommand{\dtv}{\mathrm{d}_{\mathrm{TV}}}
\newcommand{\emv}{\hat{\theta}_{\mathrm{emv}}}
\newcommand{\ent}{\mathrm{Ent}}
\usepackage{mathrsfs}
\KOMAoption{captions}{tableheading}
\makeatletter
\@ifpackageloaded{bookmark}{}{\usepackage{bookmark}}
\makeatother
\makeatletter
\@ifpackageloaded{caption}{}{\usepackage{caption}}
\AtBeginDocument{%
\ifdefined\contentsname
  \renewcommand*\contentsname{Table des matières}
\else
  \newcommand\contentsname{Table des matières}
\fi
\ifdefined\listfigurename
  \renewcommand*\listfigurename{Liste des Figures}
\else
  \newcommand\listfigurename{Liste des Figures}
\fi
\ifdefined\listtablename
  \renewcommand*\listtablename{Liste des Tables}
\else
  \newcommand\listtablename{Liste des Tables}
\fi
\ifdefined\figurename
  \renewcommand*\figurename{Figure}
\else
  \newcommand\figurename{Figure}
\fi
\ifdefined\tablename
  \renewcommand*\tablename{Table}
\else
  \newcommand\tablename{Table}
\fi
}
\@ifpackageloaded{float}{}{\usepackage{float}}
\floatstyle{ruled}
\@ifundefined{c@chapter}{\newfloat{codelisting}{h}{lop}}{\newfloat{codelisting}{h}{lop}[chapter]}
\floatname{codelisting}{Listing}
\newcommand*\listoflistings{\listof{codelisting}{Liste des Listings}}
\usepackage{amsthm}
\theoremstyle{definition}
\newtheorem{definition}{Définition}[chapter]
\theoremstyle{definition}
\newtheorem{example}{Exemple}[chapter]
\theoremstyle{plain}
\newtheorem{theorem}{Théorème}[chapter]
\theoremstyle{remark}
\AtBeginDocument{\renewcommand*{\proofname}{Preuve}}
\newtheorem*{remark}{Remarque}
\newtheorem*{solution}{Solution}
\newtheorem{refremark}{Remarque}[chapter]
\newtheorem{refsolution}{Solution}[chapter]
\makeatother
\makeatletter
\makeatother
\makeatletter
\@ifpackageloaded{caption}{}{\usepackage{caption}}
\@ifpackageloaded{subcaption}{}{\usepackage{subcaption}}
\makeatother

\ifLuaTeX
\usepackage[bidi=basic]{babel}
\else
\usepackage[bidi=default]{babel}
\fi
\babelprovide[main,import]{french}
% get rid of language-specific shorthands (see #6817):
\let\LanguageShortHands\languageshorthands
\def\languageshorthands#1{}
\usepackage{bookmark}

\IfFileExists{xurl.sty}{\usepackage{xurl}}{} % add URL line breaks if available
\urlstyle{same} % disable monospaced font for URLs
\hypersetup{
  pdftitle={TP : Apprentissage automatique et contrôle stochastique},
  pdfauthor={Samy Mekkaoui},
  pdflang={fr},
  colorlinks=true,
  linkcolor={blue},
  filecolor={Maroon},
  citecolor={Blue},
  urlcolor={Blue},
  pdfcreator={LaTeX via pandoc}}


\title{TP : Apprentissage automatique et contrôle stochastique}
\author{Samy Mekkaoui}
\date{2025-02-01}

\begin{document}
\maketitle

\renewcommand*\contentsname{Table des matières}
{
\hypersetup{linkcolor=}
\setcounter{tocdepth}{2}
\tableofcontents
}

\bookmarksetup{startatroot}

\chapter*{Organisation}\label{organisation}
\addcontentsline{toc}{chapter}{Organisation}

\markboth{Organisation}{Organisation}

\begin{center}\rule{0.5\linewidth}{0.5pt}\end{center}

Bienvenue sur la page du TP du cours d'\textbf{Apprentissage automatique
et contrôle stochastique} enseigné par Huyên Pham au sein du M2
Probabilités et Finance. Les TPs seront décomposés en 3 TPs de 3h chacun
où chacune des notions abordée pendant le cours sera illustré.

\textbf{Cette page est en cours de construction .}

\begin{itemize}
\item
  \textbf{Horaires} : Les TPs auront lieu les \(\ldots\) en salle
  \textbf{123}
\item
  \textbf{Agenda} :

  \begin{itemize}
  \tightlist
  \item
    TP n°1 : About \textbf{Reinforcement Learning}
  \item
    TP n°2 : About \textbf{Deep Galerkin} and \textbf{Deep BSDE Solver}
    for solving PDEs
  \item
    TP n°3 : About \textbf{Generative IA} and \textbf{Schrodinger
    Bridge} for data generation.
  \end{itemize}
\end{itemize}

\begin{center}\rule{0.5\linewidth}{0.5pt}\end{center}

\section*{Utiliser ce site}\label{utiliser-ce-site}
\addcontentsline{toc}{section}{Utiliser ce site}

\markright{Utiliser ce site}

Ce site est décomposé en 3 parties qui constituent le cours où un
chapitre intitulé \textbf{Course Reminders} est présenté où les
principaux résultats théoriques sont présentés ainsi qu'un second
chapitre qui contient l'énoncé du TP ainsi qu'un lien vers un fichier
jupyter notebook.

Par ailleurs, ce site est généré via \href{https://quarto.org}{Quarto}
et les notes sont accessibles depuis cette page
\href{https://github.com/SamyMekk/TP-Controle-Stochastique}{GitHub}. Si
jamais vous détectez des erreurs sur le site, n'hésitez pas à me les
faire remonter via des pull requests.

\part{Part n°1 : Reinforcement Learning}

\chapter{Course Reminders}\label{course-reminders}

\section{Some Foundations of Reinforcement
Learning}\label{some-foundations-of-reinforcement-learning}

\subsection{Markov Decision Processes}\label{markov-decision-processes}

\begin{definition}[MDP]\protect\hypertarget{def-MDP}{}\label{def-MDP}

Un intervalle de confiance de niveau \(1-\alpha\) est un intervalle
\(I = [A,B]\) dont les bornes \(A,B\) sont des statistiques, et tel que
pour tout \(\theta\), \[P_\theta(\theta \in I) \geqslant 1 - \alpha.\]
Un intervalle de confiance de niveau asymptotique \(1-\alpha\) est une
\emph{suite} d'intervalles \(I_n = [A_n,B_n]\) dont les bornes
\(A_n,B_n\) sont des statistiques, et tels que pour tout \(n\),
\[ P_\theta(\theta \in I_n) \geqslant 1 - \alpha.\]

\end{definition}

\begin{theorem}[Décomposition
biais-variance]\protect\hypertarget{thm-biaisvar}{}\label{thm-biaisvar}

Le risque quadratique \(\mathbb{E}_{\theta} [|\hat{\theta}-\theta|^2]\)
est égal à \[
\underbrace{\operatorname{Var}_{\theta} (\hat{\theta})}_{\text{variance}} +
\underbrace{\mathbb{E}_{\theta}[\hat{\theta}-\theta]^2}_{\text{carré du biais}} \, .
\]

\end{theorem}

\begin{proof}
En notant \(x\) l'espérance de \(\hat{\theta}\), on voit que le risque
quadratique est égal à
\(\mathbb{E}[|\hat{\theta} - x - (\theta - x)|^2]\). Le carré se
développe en trois termes :~le premier,
\(\mathbb{E}[|\hat{\theta} - x|^2]\), est la variance de
\(\hat{\theta}\). Le second,
\(-2\mathbb{E}[(\hat{\theta} - x)(\theta - x)]\), est égal à
\(-2(\theta - x)\mathbb{E}[\hat{\theta} - x]\), c'est-à-dire 0. Le
dernier, \(\mathbb{E}[(\theta - x)^2]\), est égal à \((\theta - x)^2\),
c'est-à-dire \((\theta - \mathbb{E}[\hat{\theta}])^2\) :~c'est bien le
carré du biais.
\end{proof}

\begin{figure}[H]

{\centering \includegraphics[width=0.5\linewidth,height=\textheight,keepaspectratio]{images/cover.png}

}

\caption{À~gauche, RQ élevé mais biais nul ;~à droite, RQ faible mais
biais non nul.}

\end{figure}%

\begin{example}[]\protect\hypertarget{exm-testadeq}{}\label{exm-testadeq}

On peut se demander si, dans la langue courante, les 21 lettres de
l'alphabet ont à peu près la même probabilité d'apparaître comme
première lettre d'un mot. Cela revient à tester si
\(\bp_0=(1/26, \dotsc, 1/26)\), hypothèse qui est évidemment fausse, il
suffit de regarder l'épaisseur des 26 sections du dictionnaire pour s'en
rendre compte.

Qu'en est-il des 9 chiffres ? On peut vouloir tester si, dans n'importe
quel document (journal, site internet, article scientifique), ces 9
chiffres apparaissent à peu près uniformément en tant que premier
chiffre d'un nombre. Cela reviendrait à tester
\(\bp_0 = (1/9, \dotsc, 1/9)\).

Ce n'est pas le cas et cette hypothèse est très fréquemment réfutée : le
premier chiffre significatif d'un nombre est bien plus souvent 1
(\(\approx 30\%\) des cas) que \(9\) (\(\approx 5\%\) cas). Ce phénomène
s'appelle \href{https://fr.wikipedia.org/wiki/Loi_de_Benford}{\emph{loi
de Benford}}.

\end{example}

\subsection{Value-based methods}\label{value-based-methods}

\subsection{Policy-based methods}\label{policy-based-methods}

\subsection{TBD}\label{tbd}

\subsection{}\label{section}

\section{Reinforcement Learning in Continous
Time}\label{reinforcement-learning-in-continous-time}

\subsection{Problem Formulation}\label{problem-formulation}

\subsection{Policy gradient methods in continous
time}\label{policy-gradient-methods-in-continous-time}

\subsubsection{Policy Gradient
Representation}\label{policy-gradient-representation}

Let's dive into policy gradient representation now.

\subsubsection{Actor critic algorithms}\label{actor-critic-algorithms}

Let's dive into Actor critic algorithms

\subsection{Q-learning and approximations in continous
time}\label{q-learning-and-approximations-in-continous-time}

\subsubsection{TBD}\label{tbd-1}

\subsubsection{TBD}\label{tbd-2}

\section{To Go Further}\label{to-go-further}

If you are interested in such topics, you can have a look at the
following papers as this is a current research topic.

\chapter{Enoncé du TP n°1}\label{enoncuxe9-du-tp-n1}

\part{Part n°2 : DeepPDE}

\chapter{Course Reminders}\label{course-reminders-1}

import numpy as np np.arange(5)

\chapter{Enoncé du TP n°2}\label{enoncuxe9-du-tp-n2}

\href{ProjetXVA.ipynb}{Télécharger le notebook pour le TP n°2}

\part{Part n°3 : Generative IA}

\chapter{Course Reminders}\label{course-reminders-2}

\section{Rappels de cours}\label{rappels-de-cours}

\begin{definition}[intervalle de
confiance]\protect\hypertarget{def-ic}{}\label{def-ic}

Un intervalle de confiance de niveau \(1-\alpha\) est un intervalle
\(I = [A,B]\) dont les bornes \(A,B\) sont des statistiques, et tel que
pour tout \(\theta\), \[P_\theta(\theta \in I) \geqslant 1 - \alpha.\]
Un intervalle de confiance de niveau asymptotique \(1-\alpha\) est une
\emph{suite} d'intervalles \(I_n = [A_n,B_n]\) dont les bornes
\(A_n,B_n\) sont des statistiques, et tels que pour tout \(n\),
\[ P_\theta(\theta \in I_n) \geqslant 1 - \alpha.\]

\end{definition}

\section{Implémentation numérique}\label{impluxe9mentation-numuxe9rique}

\(\bx\)

\begin{figure}[H]

{\centering \includegraphics[width=0.5\linewidth,height=\textheight,keepaspectratio]{images/cover.png}

}

\caption{À~gauche, RQ élevé mais biais nul ;~à droite, RQ faible mais
biais non nul.}

\end{figure}%

\chapter{Enoncé du TP n°3}\label{enoncuxe9-du-tp-n3}

\bookmarksetup{startatroot}

\chapter*{References}\label{references}
\addcontentsline{toc}{chapter}{References}

\markboth{References}{References}

\phantomsection\label{refs}
\begin{CSLReferences}{0}{1}
\end{CSLReferences}

\subsection*{Reinforcement Learning :}\label{reinforcement-learning}
\addcontentsline{toc}{subsection}{Reinforcement Learning :}

\begin{itemize}
\item
  {[}3{]} Y. Jia and X.Y. Zhou: Policy gradient and Actor-Critic
  learning in continuous time and space: theory and algorithms, 2022,
  Journal of Machine Learning and Research.
\item
  {[}4{]} Y. Jia and X.Y. Zhou: q-Learning in continuous time, 2023,
  Journal of Machine Learning and Research.
\item
  {[}6{]} R. Sutton and A. Barto: Introduction to reinforcement
  learning, second edition 2016,
\end{itemize}

\subsection*{Deep PDE :}\label{deep-pde}
\addcontentsline{toc}{subsection}{Deep PDE :}

\begin{itemize}
\item
  TBD
\item
  TBD
\item
  {[}1{]} M. Germain, H. Pham, X. Warin: Neural networks-based
  algorithms for stochastic control and PDEs in finance, Machine
  Learning and Data Sciences for Financial Markets: a guide to
  contemporary practices, Cambridge University Press, 2023, Editors: A.
  Capponi and C. A. Lehalle
\end{itemize}

\subsection*{Generative IA :}\label{generative-ia}
\addcontentsline{toc}{subsection}{Generative IA :}

\begin{itemize}
\item
  {[}2{]} M. Hamdouche, P. Henry-Labordère, H. Pham: Generative modeling
  for time series via Schrödinger bridge, 2023.
\item
  {[}5{]} C. Remlinger, J. Mikael, R. Elie: Conditional loss and deep
  Euler scheme for time series generation, 2021, AAAI Conference on
  Artificial Intelligence.
\item
  {[}7{]} M. Xia, X. Li, Q. Shen, T. Chou: Squared Wasserstein-2
  distance for efficient reconstruction of stochastic differential
  equations, 2024, arXiv:2401.11354
\end{itemize}




\end{document}
